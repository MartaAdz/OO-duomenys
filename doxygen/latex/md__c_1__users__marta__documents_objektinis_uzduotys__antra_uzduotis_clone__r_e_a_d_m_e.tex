\subsection*{Naudojimosi instrukcija}

Pasileidziame programa.


\begin{DoxyEnumerate}
\item Reikia sugeneruoti faila. Programa paklausia, kiek studentu noresime generuoti.
\item Programa informuoja, kad faile \char`\"{}kursas.\+dat\char`\"{} issaugoti pazangus studentai. T.\+y. tokie studentai, kuriu galutinis balas (naudojant vidurki) yra daugiau nei 5.
\item Programa paraso, kiek laiko uztruko programos vykdymas\+:
\end{DoxyEnumerate}
\begin{DoxyItemize}
\item studentu konteinerio sukurimas, nuskaitymas is failo,
\item isrikiavimas pagal egzamino pažymį,
\item suskirstymas i dvi grupes
\item pazangiu studentu irasymas i faila
\end{DoxyItemize}

\subsection*{Tyrimas}

{\itshape Studentu failu generavimo laikai}

\tabulinesep=1mm
\begin{longtabu} spread 0pt [c]{*{6}{|X[-1]}|}
\hline
\rowcolor{\tableheadbgcolor}\textbf{ Elementu skaicius  }&\multicolumn{5}{p{(\linewidth-\tabcolsep*6-\arrayrulewidth*2)*5/6}|}{\cellcolor{\tableheadbgcolor}\textbf{ }}\\\cline{1-6}
\endfirsthead
\hline
\endfoot
\hline
\rowcolor{\tableheadbgcolor}\textbf{ Elementu skaicius  }&\multicolumn{5}{p{(\linewidth-\tabcolsep*6-\arrayrulewidth*2)*5/6}|}{\cellcolor{\tableheadbgcolor}\textbf{ }}\\\cline{1-6}
\endhead
100  &0.\+004045  &0  &0.\+008324  &0  &0.\+003672   \\\cline{1-6}
1000  &0.\+037601  &0.\+050548  &0.\+03591  &0.\+018759  &0.\+014008   \\\cline{1-6}
10000  &0.\+377501  &0.\+255678  &0.\+303955  &0.\+227008  &0.\+307057   \\\cline{1-6}
100000  &4.\+13496  &2.\+73485  &2.\+64974  &2.\+43884  &2.\+67078   \\\cline{1-6}
\end{longtabu}



\begin{DoxyItemize}
\item Naudojant class programa vykdoma greiciau.
\item Naudojant optimizavimo flag\textquotesingle{}us, programos vykdymo laikas sokinejo, nebuvo vieningos tendencijos. \+:D \+:D \+:D
\end{DoxyItemize}

\subsubsection*{Versija v0.\+1}

\paragraph*{Reikalavimai}


\begin{DoxyEnumerate}
\item {\ttfamily develop} branch\textquotesingle{}as.
\item Palyginti {\ttfamily struct} ir {\ttfamily class} realizacijų spartą.
\item Analizė priklausomai nuo kompiliatoriaus optimizavimo lygio, nurodomo per flag\textquotesingle{}us.
\item {\ttfamily master} ir {\ttfamily develop} merge\textquotesingle{}as.
\end{DoxyEnumerate}

\paragraph*{Realizacija}


\begin{DoxyEnumerate}
\item {\ttfamily struct} perdaryta į {\ttfamily class}. Palyginta sparta.
\item Atliktas tyrimas su optimizavimo flag\textquotesingle{}ais.
\item Darbas su branch\textquotesingle{}ais. Sukurtas {\ttfamily develop} ir padarytas merge su {\ttfamily master}.
\end{DoxyEnumerate}

\subsubsection*{Versija v0.\+2}

\paragraph*{Reikalavimai}


\begin{DoxyEnumerate}
\item doxygen dokumentacija
\item operatoriu overload\textquotesingle{}inimas
\end{DoxyEnumerate}

\paragraph*{Realizacija}


\begin{DoxyItemize}
\item Idetas {\ttfamily stable\+\_\+partition} algoritmas
\item Pašalinta kirstymo funkcija
\end{DoxyItemize}

\subsubsection*{Versija v1.\+0}

\paragraph*{Reikalavimai}


\begin{DoxyItemize}
\item Unit testai
\end{DoxyItemize}

\paragraph*{Realizacija}


\begin{DoxyItemize}
\item Padaryti unit testai naudojant Catch2 framework\textquotesingle{}a
\end{DoxyItemize}

\subsubsection*{Cool dalykai}


\begin{DoxyEnumerate}
\item {\ttfamily Class} iskart greitesne nei {\ttfamily struct}
\item 
\begin{DoxyCode}
\mbox{\hyperlink{classstudentas_a40a99ea5d527a3d443123f4785550787}{studentas::studentas}}(std::istream&duomenys)\{
    stud\_fromFile(duomenys);
\}
\end{DoxyCode}
 3. 
\begin{DoxyCode}
\textcolor{keywordtype}{int} main()
\{
        \textcolor{keywordtype}{unsigned} \textcolor{keywordtype}{int} stud\_kiekis;
        stud\_kiekis = 10000;
        \textcolor{keyword}{const} \textcolor{keywordtype}{int} paz\_kiekis = 5;

        studentu\_generavimas(stud\_kiekis, paz\_kiekis);

        std::ifstream duomenys(\textcolor{stringliteral}{"kursiokai.txt"});

        \textcolor{keywordflow}{try}\{
            \textcolor{keywordflow}{if}(duomenys.fail())
            \{
                \textcolor{keywordflow}{throw} std::exception();
            \}
        \}
        \textcolor{keywordflow}{catch} (std::exception &e1)
        \{   e1.what();
            std::cerr<<\textcolor{stringliteral}{"Tokio failo nera. Baigiu programa."};
            exit(1);
        \}

            std::vector<studentas> S;
            visi\_toVec(duomenys, S, stud\_kiekis);
            rikiavimas\_vec(S);
            dvi\_grupes(S);
            stud\_toFile\_vec(S);

        \textcolor{keywordflow}{return} 0;
 \}
\end{DoxyCode}

\end{DoxyEnumerate}

\section*{Pastabeles}


\begin{DoxyItemize}
\item Doxygen nepilnas
\item Dvejojame kada reikia \char`\"{}dieviskosios trejybes\char`\"{}
\item Neveikia Unit testas 
\end{DoxyItemize}